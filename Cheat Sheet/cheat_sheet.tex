\documentclass[10pt,landscape]{article}
\usepackage{multicol}
\usepackage{calc}
\usepackage{setspace}
\usepackage{ifthen}
\usepackage[landscape]{geometry}
\usepackage{graphicx}
\usepackage{listings}

\geometry{top=.2in,left=.2in,right=.2in,bottom=.2in} 
\usepackage{lmodern}
\usepackage{listings}
\lstset{language=C,
numberstyle=\footnotesize,
basicstyle=\footnotesize,
stepnumber=1,
breaklines=true}
%\usepackage[subsection]{placeins}
\usepackage{float}

\pagestyle{empty}

\lstset{basicstyle=\footnotesizev,breaklines=true}
\lstset{framextopmargin=50pt}

\makeatletter
\renewcommand{\section}{\@startsection{section}{1}{0mm}%
                                {-1ex plus -.5ex minus -.2ex}%
                                {0.5ex plus .2ex}%x
                                {\normalfont\large\bfseries}}
\renewcommand{\subsection}{\@startsection{subsection}{2}{0mm}%
                                {-1explus -.5ex minus -.2ex}%
                                {0.5ex plus .2ex}%
                                {\normalfont\normalsize\bfseries}}
\renewcommand{\subsubsection}{\@startsection{subsubsection}{3}{0mm}%
                                {-1ex plus -.5ex minus -.2ex}%
                                {1ex plus .2ex}%
                                {\normalfont\small\bfseries}}
\makeatother

\def\BibTeX{{\rm B\kern-.05em{\sc i\kern-.025em b}\kern-.08em
    T\kern-.1667em\lower.7ex\hbox{E}\kern-.125emX}}

\setcounter{secnumdepth}{0}

\setlength{\parindent}{0pt}
\setlength{\parskip}{0pt plus 0.5ex}

\newcommand{\mysinglespacing}{%
  \setstretch{1}% no correction afterwards
}

\lstset{
         basicstyle=\footnotesize\ttfamily, 
         numberstyle=\tiny,          
         numbersep=2pt,             
         tabsize=2,                
         extendedchars=true,      
         breaklines=true,        
         showspaces=false,      
         showtabs=false,       
         xleftmargin=0pt,
         framexleftmargin=0pt,
         framexrightmargin=0pt,
         framexbottommargin=0pt,
         showstringspaces=false 
         basicstyle=\tiny
 }

\lstloadlanguages{
         C
 }

\begin{document}



% Using Courier font
\renewcommand{\ttdefault}{cmtt}

\raggedright
\footnotesize
\begin{multicols}{3}

\setlength{\premulticols}{1pt}
\setlength{\postmulticols}{1pt}
\setlength{\multicolsep}{1pt}
\setlength{\columnsep}{2pt}

\begin{center}
     \Large{\textbf{Systems Programming Cheat Sheet}} \\
\end{center}

\section{Mutex locks}

A \emph{mutex} is a lock that we set before using a shared resource and release after using it. 
When the lock is set, \emph{no other thread can access the locked region of code}. 
So we see that even if thread 2 is scheduled while thread 1 was not done accessing 
the shared resource and the code is locked by thread 1 using mutexes then thread 2 
cannot even access that region of code. So this \emph{ensures a synchronized access of shared resources in the code}.

\lstinputlisting[language=C]{snippets/mutex.c}

\section{Signal locks}
\section{Signal handling}

A {\bf signal} is a condition that may be reported during program execution, 
and can be ignored, handled specially, or, as is the default, 
used to terminate the program.

\lstinputlisting[language=C]{snippets/signal.c}

\section{Multiprogramming}

\subsection{Fork}
\subsection{Exec}
\subsection{Wait}

\section{Shell scripting}
\section{Shared memory}
\section{Pointers to functions}

\lstinputlisting[language=C]{snippets/funcpoint.c}

\section{Deadlock}



\section{Semaphores}

A \emph{semaphore} is a special type of variable that can be incremented or decremented, 
but \emph{crucial access to the variable is guaranteed to be atomic}, even in a multi-threaded program.
If two or more threads in a program attempt to change the value of a semaphore, 
the system guarantees that all the operations will in fact \emph{take place in sequence}.

\end{multicols}
\end{document}
